\begin{center}
{\bf Abstract}
\end{center}
\noindent
Quantum fluids possess amazing properties of which two are particularly striking. Firstly they exhibit superfluid flow, with the total absence of viscosity. Secondly, there are no excitations when the fluid velocity (relative to some obstacle) is slower than a critical value; above this velocity the flow becomes dissipative and macroscopic excitations are created in the form of quantised vortices with fixed circulation proportional to Planck's constant. In this thesis we numerically study the dynamics of these phenomena, from the production of a single vortex pair to the complex and chaotic motion of turbulent vortex tangles, modelling both superfluid helium and atomic Bose-Einstein condensates (BEC) with various shaped obstacles. We give detailed descriptions of the numerical schemes and present extensive numerical simulation of the Gross-Pitaevskii equation (GPE) and its variants at zero temperature and beyond, in both two and three dimensions.

We study the wake that forms behind obstacles in the presence of a superfluid flow, modelling atomic BEC experiments with moving laser-induced potentials. We find that suitable obstacles produce classical-like wakes consisting of clusters of vortices of the same polarity. Remarkably, symmetric wakes resemble those observed in classical flow at low Reynolds number, despite the constrained vorticity.  The structures are unstable, forming time-dependent asymmetric wakes similar to a B\'enard--von K\'arm\'an vortex street.

Motivated by the recent work of Kwon {\it et al.} (Phys. Rev. A {\bf 90}, 063627
(2014)), we model an atomic BEC experiment in which a trapped, oblate condensate is translated past a stationary, laser-induced obstacle. The critical velocity is exceeded and so vortices nucleate, forming a state of two-dimensional quantum turbulence. We explore the system at both zero-temperature and with thermal dissipation, modelled through a phenomenological term in the GPE. Our simulations provide insight into early-stage evolution, not accessible experimentally, and into the decay of vortices by annihilation or passage out of the condensate.

We use classical field methods to simulate homogeneous Bose gases at finite temperature, from strongly non-equilibrium initial distributions to thermalised equilibrium states. We introduce a moving cylindrical potential and study how the thermal component of the gas affects vortex nucleation. We have found that the critical velocity decreases with increasing temperature and scales with the speed of sound. Above the critical velocity, vortices are nucleated as irregular vortex lines, rings, or vortex tangles.

Finally we model the surfaces of walls and moving objects (wires, grids, propellers, spheres) with the flow of superfluid liquid helium using a real rough boundary, obtained via atomic force microscopy. We find evidence pointing to the formation of a thin `superfluid boundary layer' consisting of 
vortex loops and rings. As boundary layers usually arise from viscous forces, this is a surprising and intriguing result.
\thispagestyle{empty}
\cleardoublepage
