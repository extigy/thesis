\begin{abstract}
An intriguing feature of quantum fluids is lack of excitations when the fluid velocity is slower than a critical value; above this velocity the flow becomes dissipative and, due to a direct consequence of quantum mechanical effects, macroscopic excitations are created in the form of quantised vortices with fixed circulation $\kappa$, proportional to Planck's constant. Quantum fluids also exhibit superfluidity; flow with the absence of viscosity. Recent experimental, numerical and theoretical studies have highlighted unexpected and remarkable similarities between turbulence in quantum fluids (consisting of the motion of many quantised vortices) and turbulence in ordinary classical fluids, despite the lack of viscosity and constrained vorticity.

In this thesis we quantitatively and qualitatively study the dynamics of these phenomena, from the production of a single vortex pair to the complex and chaotic motion of turbulent vortex tangles, in both superfluid Helium II and atomic Bose-Einstein condensates. We model the quantum fluids both at zero temperature and beyond through use of the Gross-Pitaevskii equation and its variants, in both two and three dimensions. Extensive numerical simulation of the Gross-Pitaevskii equation is performed, providing insight into the dynamics of quantum fluids in the presence of quantum turbulence.

We study the wake that forms behind various obstacle shapes in the presence of a superfluid flow, modelling ultra low temperature atomic BEC experiments.

We study the decay of quantum turbulence, generated by a large obstacle in an atomic BEC.

We consider a Bose gas at finite-temperature, and discuss how the thermal component affect vortex nucleation. 

Finally we study the effect of rough surfaces on superfluid flow, modelling the vibrating wire experiments in Helium II.


\end{abstract}
\thispagestyle{empty}
\cleardoublepage
