\begin{chapter}{\label{cha:theoretical_model}Theoretical Modeling of BEC}
\section{\label{section:meanfield} Mean-field description}
\section{\label{section:gpe} The Gross-Pitaevskii Equation}
	\begin{equation}
	\mathrm{i} \hbar \frac{\partial\Psi({\bf r},t)}{\partial t} = \left(-\frac{\hbar^2}{2m}\nabla^2 + V({\bf r},t) + g|\Psi({\bf r},t)|^2 - \mu \right) \Psi({\bf r},t).
	\label{eq:gpe}
	\end{equation}
	Where $V({\bf r},t) = V_{\mathrm{obj}}({\bf r},t) + V_{\mathrm{trap}}({\bf r},t)$. When trapped, the trap is harmonic and of the form $V_{\mathrm{trap}}=m\omega^2r^2/2$, otherwise $V_{\mathrm{trap}}=0$.
\section{\label{section:quasi2dgpe} Quasi-Two-Dimensional Gross-Pitaevskii Equation}
	When $\omega_z >> \omega_r$ and $\hbar\omega_z >> \mu$ the condensate becomes highly oblate. Tight $z$ confinment causes the dynamics to become essentially two dimentional.
	In this case a 2DGPE can be used to model the system where $g_{2D} = g/\left( \sqrt{2\pi}l_z\right )$. [CITE PARKER THESIS] The chemical potential is also modified. See Section \ref{section:mu} for details on $\mu$.

\section{\label{section:gpedimless} Dimensionless Gross-Pitaevskii Equations}
	\subsection{\label{section:gpedimlesshomg} Homogeneous GPE}
		When discussing a homogeneous condensate we drop the dimensionless modifiers for each quantity and use the equation,
		\begin{equation}\label{eq:dimgpehomg}
		\mathrm{i}\frac{\partial\psi({\bf r},t)}{\partial t} = \left( -\frac{1}{2}\nabla^2 + |\psi({\bf r},t)|^2 + V_{\mathrm{obj}}({\bf r},t) - 1 \right) \psi({\bf r},t).
		\end{equation}
	\subsection{\label{section:gpedimlesstrap} Trapped GPE}
		When discussing the a trapped condensate we drop the dimensionless modifiers for each quantity and use the equation,
		\begin{equation}\label{eq:dimgpetrapped}
		\mathrm{i}\frac{\partial\phi({\bf r},t)}{\partial t} = \left( -\frac{1}{2}\nabla^2 + g|\phi({\bf r},t)|^2 + V({\bf r},t) - 1 \right) \phi({\bf r},t).
		\end{equation}
		Where $V({\bf r},t) = \frac{r^2}{2} + V_{\mathrm{obj}}({\bf r},t)$

\section{\label{section:gpe} The Dissipative Gross-Pitaevskii Equation}
	\subsection{\label{section:gamma} Phenomenological dissipation}
	The GPE can be modified to provide a simple phenomenological model of a condensate's interaction with the thermal cloud. The phenomenological damping term, $\gamma$, is added to the right hand side of the GPE with the effect that the energy in the system no longer remains constant. The energy will instead vary over time to approach some constant value. This has the effect of damping out any excitations made to the condensate, and over time the wavefunction approaches the steady state. A microscopic justification for this model was provided by Penckwitt et al [CITE] and Gardiner at al [CITE]; by studying the growth of a condensate in the presence of a rotating thermal cloud an expression for $\gamma$ was found.
		\begin{equation}\label{eq:dissgamma}
		\gamma = \frac{4m\tilde{g}a^2kT}{\pi\hbar^2} \approx 0.01,
		\end{equation}
	where $k$ is Boltzmann's constant and $\tilde{g} = 3$ is a factor used for correction. As $\gamma$ is proportional to temperature, in this thesis various values of $\gamma$ will be used as a qualitative probe of finite-temperature dynamics with only marginally more complex numerical methods.
	In the case with a homogeneous condensate this leaves us with
		\begin{equation}\label{eq:dissgpehomg}
		(\mathrm{i} - \gamma)\frac{\partial\psi({\bf r},t)}{\partial t} = \left( -\frac{1}{2}\nabla^2 + |\psi({\bf r},t)|^2 + V_{\mathrm{obj}}({\bf r},t) - 1 \right) \psi({\bf r},t),
		\end{equation}
	and in the case with a trapped condensate this leaves us with
		\begin{equation}\label{eq:dissgpetrapped}
		(\mathrm{i}-\gamma)\frac{\partial\phi({\bf r},t)}{\partial t} = \left( -\frac{1}{2}\nabla^2 + g|\phi({\bf r},t)|^2 + V({\bf r},t) - 1 \right) \phi({\bf r},t).
		\end{equation}

	\subsection{\label{section:mu} The role of the chemical potential}
\section{\label{section:hydrodynamic} Hydrodynamic interpretation}
	Often it can be helpful to write the GPE, via the so called Madelung transformation, as a set of hydrodynamic equations. The transformation reinterprets the wavefunction $\Psi$ as a magnitude directly related to the fluid density and a phase which is directly related to the fluid velocity. We write the wavefunction in the form
	\begin{equation}
		\Psi({\bf r},t) = R({\bf r},t)\exp (\mathrm{i}\theta({\bf r},t)),
	\end{equation}
	 and identify the fluid density as $\rho=mR^2$ and the velocity as $\mathbf{v} = \frac{\hbar}{m}\nabla\theta$.
	In vector form we obtain a continuity equation
	\begin{equation}
	  \frac{\partial \rho}{\partial t} + \nabla(\rho{\bf v}) = 0,
	  \label{eq:MTcont}
	\end{equation}
	and an equation similar to the Euler equation for an inviscid fluid,
	\begin{equation}
	\rho\left( \frac{\partial \mathbf{v}}{\partial t} + \left( \mathbf{v} \cdot \nabla \right)\mathbf{v} \right) = -\nabla p - \nabla \mathbf{P} - \rho \nabla \left(\frac{V}{m}\right).
	\end{equation}
	where $P_{jk} = -\frac{\hbar^2}{4m^2}\rho\frac{\partial^2\ln{\rho}}{\partial x_j \partial x_k}$.
	A detailed derivation of this result can be found in Appendix \ref{appsection:madtrans}.

\section{\label{section:solutions} A selection of analytical solutions}
	\subsection{\label{section:wall} Density near a wall}
	\subsection{\label{section:soliton} Soliton solutions}
		\begin{equation}
		\Psi(x) = \Psi_0 \tanh \left( \frac{x}{\sqrt{2}\xi} \right)
		\label{eq:soliton}
		\end{equation}
	\subsection{\label{section:vortices} A note on vortex solutions}
\section{\label{section:inital} Initial Conditions}
	\subsection{\label{section:tftrap} Thomas Fermi profile of a trapped condensate}
	Fixed in time $\Psi$ and $V$.
	\begin{equation}
	\sqrt{\frac{\mu - V({\bf r})}{g}} =  \Psi({\bf r})
	\label{eq:TF}
	\end{equation}
	\subsection{\label{section:cfield} Classical Field approximation with a homogeneous condensate}
		\begin{equation}
		\psi({\bf r},t) = \sum_{\bf k} a_{\bf k} \exp (\mathrm{i}{\bf k}\cdot{\bf r}),
		\label{eq:cFieldIC}
		\end{equation}
		where the complex Fourier amplitudes $a_{\bf k}$ are related to the occupation numbers $n_{\bf k}$ through $\braket{a_{\bf k}^{\,}a_{\bf k'}^*} = n_{\bf k}\delta_{\bf kk'}$. The phase of the complex amplitudes $a_{\bf k}$ are distributed uniformly on $[0,2\pi]$ while $|a_{\bf k}|$ is distributed randomly with fixed mean equal to unity; it has been found that different distributions of $|a_{\bf k}|$ make no qualitative difference to the turbulent evolution[phys rev A 66 013603]. [TODO: write about choice of E and N here to get different condensate fractions]

		We also have the integral distribution function,
		\begin{equation}
		D_k = \sum_{k'<k}n_{\bf k}.
		\label{eq:intDistFunc}
		\end{equation}
		This is a coarse-grained characteristic of the particle distribution which shows how many particles have momenta less than $k$.
\end{chapter}
